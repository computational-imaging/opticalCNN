\paragraph{Convolutional neural networks.} 
Artificial neural networks were proposed in X. Early networks were composed of fully connected layers with nonlinear activation functions in between, inspired by the canonical biological neuron and its thresholded activation. Convolutional layers were popularized by LeCun and ... in image classification CITE. Convolutional layers allow for weight sharing... Since then, deeper, more complex, etc. 

As embedded vision and even continuous mobile vision become 
hardware
incorporation of image processing on the sensor chip, eliminating or reducing the need to shuttle full image data to a processor. These chips have been designed to detect edges and orientations and to perform wavelet or discrete cosine transforms \cite{gruev2002implementation} [RedEye]. Most of these approaches still rely on electronic computation on the image sensor chip, whereas our goal is all-optical implementation with no additional power input. 

\paragraph{CNN architecture variations}

Our goal is to match performance with a constrained optical setup, so also relevant to highlight are CNNs with non-standard architectures that may align with physical designs. Omission of fully connected layers, i.e. fully convolutional with global average pooling at the top layer has proven to be successful in \cite{lin2013network,iandola2016squeezenet}. Analysis of CNN operations in the Fourier domain, introducing spectral pooling and regularization \cite{rippel2015spectral}. Relevant because we can also access optical Fourier plane. We also note the work in the complex-valued deep neural networks \cite{trabelsi2017deep}, as coherent optical signals may be an effective means of propagating complex-valued data.
 

\paragraph{Optical computing and computational light transport.}
In the computational imaging community, many new optical system designs exploit the physical propagation of light to 
The co-design of optics and algorithms 

Optical computing offers high bandwidth, but high cost. Optoelectronics and fully optical. Optical solutions to NP-complete problems that are faster than electronic computation \cite{wu2014optical}.

% Computational photography has some intersection with optical computing in that they may perform some operations on the input signal optically, but they are also distinct in that they work with spatially organized inputs that come from physical world (incoherent light). Coded apertures and PSF engineering can perform filtering [CITE]. Optical correlators that essentially perform template matching on images have been explored for optical target detection and tracking \cite{manzur2012optical, javidi1995optical}.  

\paragraph{Optical neural networks.}   The concept of an optical neural network (ONN) captured the attention of many in the late 1980s to mid-1990s, primarily due to the capability of optics to perform the expensive matrix multiply of a fully connected layer. In 1985, an optoelectronic implementation of the Hopfield model, a basic model of a recurrent neural network, was created with one-dimensional (1D) LED array input signals and a binary transmission mask \cite{farhat1985optical}. This model divided the weight matrix into two parts, positive and negative, and required electronics for subtraction of the two parts and signal thresholding. Psaltis et al. further explored the potential of dynamic photorefractive crystals to store neural network weights, which could allow for optical backpropagation-based learning in ONNs \cite{psaltis1988adaptive}. Meanwhile, the optoelectronic network of a Hopfield model was extended to 2D signals by partitioning the pixels of a liquid crystal television to store an array of smaller 2D patterns \cite{lu1989two}. Furthermore, an optical thresholding perceptron was implemented with liquid crystal light valves (LCLV), which disposed of the need to convert between optical and electronic signals between layers \cite{saxena1995adaptive}. A more extensive overview of the varied implementations of ONNs can be found in \cite{denz2013optical}.

Despite the accumulation of insights in this area, as neural networks fell out of the spotlight, the demand for ONNs also waned. However, with the resurgence of CNNs that are far more powerful and computationally expensive than before, there is renewed interest in optical computing \footnote{Fathom Computing (\url{fathomcomputing.com}), Lightelligence (\url{lightelligence.ai}), Optalysis (\url{optalysys.com)}}. Recent works that connect efforts of the last century to modern hardware include a two-layer fully connected neural network based on programmable photonic circuits \cite{shen2017deep} and  a recurrent neural network with DMD-based weights \cite{bueno2017reinforcement}. However, none of the ONNs mentioned previously involve convolutional layers, which have become essential in computer vision applications. The ASP Vision system approaches the task of designing a hybrid optoelectronic CNN, using angle sensitive pixels to approximate the first convolutional layer of a typical CNN, but it is limited to a fixed set of convolution kernels \cite{chen2016asp}. Our goal is to design a system with optimizable optical elements to demonstrate low-power inference by a custom optical or optoelectronic CNN.

