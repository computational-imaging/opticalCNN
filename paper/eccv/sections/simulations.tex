We use simulations to better understand the performance of optical CNNs. \\
\red{Fig. 2}: diagram of possible ONN models.
\red{Table 1:} Results.
 Introduce toy classification problem(s?), discuss constraints.

\subsection{Learned Optical Correlator}
For our first experiment, we simulated a system with a single optical convolutional layer to confirm that our proposed optical convolution layer would function as expected. A single convolutional layer is essentially an optical correlator.  

Here we are able to use end-to-end learning. \\


\red{Possible figure with learned phase mask and PSF.}

While this was interesting, optical correlator is not powerful enough for more difficult classification tasks, for example with natural images or with more categories. Also, with a single layer, it was not necessarily a CNN.

\subsection{Hybrid optoelectronic CNN} 
Next we keep one optical convolutional layer but add on more after. 
\red{Fig. 3: Hybrid ONN phase masks and PSFs.}
	Grayscale \\

\subsubsection{Pseudo-negative weights}
	Talk about the dual channel positive and negative weights\\


\subsubsection{Color filters - Vincent}



\subsection{Fully optical CNN} 
Doesn’t fully work, but can discuss some results